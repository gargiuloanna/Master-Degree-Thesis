%cap 1 descrizione generale del problema delle cadute e della prevenzione nell'ambito medico, ma non ingegneristico.
\section{Problem Definition}
The topic of this thesis revolves around the study of the \textit{prevention} of falls among adults and elderly. More specifically, the study is based on a wearable sensor platform, whose sensors give an insight on the the way the patient moves, to later estimate their risk of falling. 
Although several research studies have been published regarding the \textit{detection} of falls using different methodologies, fall prevention remains an hot topic in the scientific community. The main difference between prevention and detection of falls is based on the idea that the first aims at identifying the risk of falling of a patient, whereas the latter aims at recognizing a fall that has already occurred.

With the ever growing scientific advances and the longer life expectancy, falls have become one of main causes of injury among the elderly. 
Injuries include fractures, brain damage, mobility and independence loss, and even death. According to the Centers for Disease Control and Prevention (CDC), more than one in four older adults report a fall each year. 

In patients affected by Parkinson's Disease, a common problem related to the risk of falls is the Freezing of Gait (FOG). In determining whether or not this parameter could be predicted, it has been demonstrated that utilizing EEG or EMG signals led to a more accurate prediction. 

As stated by \cite{ImpactFallNurses}, falls not only impact the patients themselves, but also their surroundings. Psychological effects take place on nurses, who feel a higher level of stress and pressure when dealing with a patient who is at risk of falling. This results in scenarios in which a nurse would limit the mobility and individual freedom of the patient as to avoid the consequences of an involuntary fall. This leads to, as stated by the nurses themselves, the loss of strength in the patients, who also have their privacy taken away: to avoid falls, nurses even accompany them in the bathroom, since a fall would get them in trouble and slow their work down. 
In this scenario, a different approach to fall prevention is necessary.

\subsection{Fall Risks}
The prevention of falls relies in the identification of risks associated with an increased chance of falling, and then on the mitigation of the fore mentioned risks.
A study published in 2022 \cite{RiskFactors} compared all the causes that were thought to be relevant in the context of falls, and evaluated among those that increase the risk of falling in the elderly:
\begin{description}
   \item[Older Age] Falls are the most common cause for trauma in older patients. Injuries at age above 65 lead to worse head injuries, in particular hemorrhage, compared to the younger counterpart \cite{geriatricTrauma}.
   \item[Frailty] The elderly is generally considered more frail than their younger counterpart. Frailty is defined as the natural decline of the functions among organs \cite{geriatricTrauma}.
   \item[Drugs Used] Psychotropic and psychoactive drugs may increase the risk of falls in a skilled nursing facility in proportion to the total load of these agents \cite{drugsEffects}.
   \item[Polypharmacy] Polypharmacy is defined as the use of multiple medicines, and it is associated with adverse outcomes 
   including mortality, falls, adverse drug reactions, increased length of stay in hospital and readmission to hospital soon 
   after discharge \cite{PolypharmacyRisk}.
   \item[Heart Disease] More than 60\% of patients affected by any kind of heart complication have a moderate to high risk for falls. Adults with Heart Failure and arrhythmias have an especially high risk of falls (likely because of diminished cardiac output, polypharmacy, or interaction with other comorbid conditions), with a fall rate of 13\% higher if compared to people with other chronic diseases \cite{HeartRisk}.
   \item[Hypertension] This risk is not directly associated with the condition itself, but rather, it is related to the use of medication to control it. As stated beforehand, the use of drugs and polypharmacy are associated with an increased risk of falling, thus any pathology that leads to the addition of more medication is to relate to a risk of fall. 
   \item[Fall History] A study \cite{HistoryRisk} demonstrated that a number of previous falls up to 4 or 5 is a predominant factor in determining whether an elderly is at risk of falling or not. The research is supported by the evidence that while a single fall might occur at random, recurrent fallers are likely to suffer from persistent deficits that result in an inability to avoid falls.
   \item[Depression] Both depression and the use of antidepressant increase the risk of fall. This is due to different reasons: depression increases fall risk through psychomotor retardation, deconditioning, gait/balance abnormalities, impaired sleep/attention and fear of falling; instead, antidepressants contribute to (or cause) falling through causing sedation, impaired balance/reaction time, OH, hyponatremia, cardiac conduction delay/arrhythmia, and/or drug-induced Parkinsonism \cite{DepressionRisk}.
   \item[Parkinson's Disease] Fall rates are higher in people affected by Parkinson's Disease (35\% to 90\% of patients fell at least once, with an average rate of 60.5\%). Historically, falls were seen as a late manifestation of the disease caused by a progression of axial motor problems combined with the effects of aging. In this context, recurrent falling is a milestone in disease progression often linked to other events such as visual hallucinations, cognitive decline, and hospitalization \cite{ParkinsonRisk}.
   \item[Pain] A research paper \cite{PainRisk}, demonstrated that the elderly subject to mild, moderate and severe pain in more that two body sites were associated with an higher risk of falling compared to those who had not, as stated by the follow up questions about history of falls during the course of two years.
\end{description}

Some other risk factors comprise:
\begin{description}
  \item[Malnutrition] The topic of the correlation between malnutrition and fall risk is still highly discussed in the scientific community. Most research focus on other conditions to determine the risk, hence less studies investigated the relationship between the two. Kupisz-Urbanska and Marcinowska-Suchowierska's review \cite{MalnutritionRisk} concluded that malnutrition is indeed a factor of risk, due to the loss of muscle strength, mass and function, which also leads to more harmful falls.
  \item[Living Alone] The New York Times \cite{AloneRisk} reports that people who live alone in their sixties are 24\% more likely to fall than those with greater social interactions. Even for hospitalization, the rate is 23\% higher in people living alone and 36 percent higher among those with the least social contact compared with those with the most.
  \item[Living in a rural area] There are different theories regarding whether or not living in a rural area is an increased risk for falling. Different results were obtained by different studies, which confirmed or denied any correlation between the two. Mostly, a fall in rural area is to be attributed to both the working conditions and location of the work, which exposes the employer to dangerous environments and movements.
  \item[Alcohol Consumption] A study on alcohol consumption in Chinese adults \cite{AlcoholRisk} highlighted that usual alcohol consumption is associated with a higher risk of falls, since it can interfere with balance, coordination, and vision.
\end{description}

Some other conditions related to the risk of falling that were not demonstrated by \cite{RiskFactors} to increase such risk, but on which literature does not agree are:
\begin{description}
  \item[BMI] Although BMI itself may not be a risk, consequences linked to BMI might be. Excessive weight may put additional strain on joints and lead to mobility issues. Moreover, people may be in pain or using medications.
  \item[Sex] Most studies focused on fall risk identification - whether to determine if a pathology or a condition is of risk - kept into consideration biological data such as age and sex. Almost every study found a relation between sex and the increased fall risk, demonstrating that the female sex falls more often compared to their male counterparts. This result was also achieved when considering multiple conditions at once, confirming that males and females affected by the same pathology, or living in the same environments, had different fall rates.
  \item[Education level] Education level is also debated. Considerations can be made regarding the age associated to education level (e.g., children may be at higher risk of falling), or a diversity in living conditions, economic status, access to healthcare, and job mobility requirements.
  \item[Diabetes] While diabetes is not a condition that increases the fall risk itself, the consequences of this disease can lead to problems such as vision problems, hypoglicemia, and thus balance problems and muscle problems. Not to mention the medications needed to keep diabetes under control that may lead to an increased risk of falling in diabetic patients.
  \item[Stroke] A review on the state of the art (of 2016) about the incidence of falls in stroke survivors \cite{StrokeRisk} highlighted that there are contrasting opinions regarding whether or not this factor increases the chances of a fall. One reason being that more severe stroke cases may have a reduced chance of falling due to their limited mobility. Another reason being that results differ due to regional differences among stroke survivors. Moreover, some studies have limited their observations to 2 years post stroke, whereas this review confirmed from longer term studies that stroke survivors  experience falls twice as likely than non-stroke controls after a median of 10 years post-stroke. 
  \item[Vision Dysfunction] A study from 2010 \cite{VisionRisk} concluded that vision impairment caused 
  significant gait changes, since obstacle-crossing is more cautious when compared with that of visually normal subjects; 
  suggesting that such individuals may be at greater risk of tripping or falling during everyday locomotion.
  \item[Cognitive Impairment] Although the analysis carried out by \cite{RiskFactors} has not demonstrated the correlation of this factor with fall risk, \cite{CognitiveRisk} explored the steps to reduce the risk of fall in dementia patients. From their study background, people who suffer from cognitive impairment fall two to three times more than cognitively healthy older adults. The data shows that 60 to 80\% of people with dementia fall annually. The regions of the brain involved in dementia are required to coordinate mobility, balance, and gait, leading to a variability in stride length and reduced walking speed.
\end{description}

\subsection{Fall Risk and Balance Tests}
To assess the risk of falling, literature agrees that it is useful to evaluate the movements of the patients using a variety of tests. The common denominator though, is that research shows that just one of these tests is not enough to determine if the elderly is at risk, but rather a combination of factors can be used to advance a conclusion.

\vspace{0.3cm}
\textbf{Berg Balance Scale (BBS)} %citazioni ok

The Berg Balance Scale is a test made up 14 static and dynamic activities related to everyday living. The BBS assesses balance and risk for falls through direct observation of the participant's performance by trained health care professionals in a variety of settings. The patient is expected to:
\begin{itemize}
\itemsep0cm 
    \item Move from a sitting to a standing position;
    \item Stand up unsupported;
    \item Sit unsupported;
    \item Move from a standing to a sitting position;
    \item Transfer from one chair to another;
    \item Stand up with their eyes closed;
    \item Stand with their feet together;
    \item Reach forward with an outstretched arm;
    \item Pick an object up off of the floor;
    \item Turn and look behind them;
    \item Turn around in a complete circle;
    \item Place each foot alternately on a stool in front of them;
    \item Stand unsupported with one food directly in front of the other;
    \item Stand on one leg for as long as they can.
\end{itemize}
Scoring is on a 5-point ordinal scale with 0 indicating the inability to complete the task and 4 as independent with 
completing the task. The maximum score of 56 indicates good balance \cite{BergBalanceScale}, 
whereas a score below 45 indicates individuals may be at greater risk of falling. As stated before, 
nowadays this test is valid for predicting a fall only if supported by other tests in this category. 
Otherwise, it can only be an indicator for static balance. 

\vspace{0.3cm}
\textbf{Timed Up and Go Test(TUG)}

The aim of this test is to assess the patient's mobility and fall risk.
It consists of simple activities, with the objective of determining the amount of time that the subject needs to complete the tasks. 
The only equipment needed is a chair and a stopwatch.
After marking a 3 meters line on the floor starting from the chair, the patient is asked, as soon as the stopwatch is started, 
to stand up from the chair, reach the target at their normal walking pace, turn around and sit back into the chair.
The timer is then stopped, and the time is recorded, in seconds. 
\\Any older adult whose results are above or equal to 12 seconds is at risk of falling \cite{TUG}.

\vspace{0.3cm}
\textbf{Balance Evaluation Systems Test (BESTest)}

The BESTest is comprised of a variety of tasks to assess any postural complications related to a loss of balance. It is intended for patients with neurological conditions (e.g. Parkinson's Disease, Stroke, Multiple Sclerosis and so on), vestibular disorders,
cognitive impairments, and the elderly. 
The tasks are split into 6 sections for a maximum of 108 points:
\begin{description}
\itemsep0cm 
    \item[\textit{Biomechanical Constraints}] 
    This section measures the base of support (whether both feet have normal base of support or the support is subjected to pain and/or deformities), the Center of Mass Alignment, to determine if the posture is correct, ankle strength and range, hip/trunk lateral strength, the ability to sit on the floor and stand up;
    \item[\textit{Stability limits/verticality}]
    This section tests the ability to sit and lean to the side, and the capability to realign vertically after the movement. Moreover, it comprises the evaluation of the forward and lateral reach;
    \item[\textit{Transitions}]
    The activities in this section test the autonomy to stand starting from a sitting position, the stability of the keeping a balance on the toes, to stand on one leg, the ability to go up the stairs and the balance while keeping one arm raised;
    \item[\textit{Reactivity}]
    This part of the BESTest determines the overall responsiveness of the patient when pushed away from a balanced position; whether they are able to get back to a balanced positions or not. The push is tested in the forward and backward position; and also a compensatory stepping correction (how many steps the patient takes to recover equilibrium) is tested in both the forward and backward direction, with the addition of a lateral correction test;
    \item[\textit{Sensory Orientation}]
    These later activities evaluate the impact other senses have on the overall balance of the patient. In particular, they test motions the subject is asked to perform with both open and closed eyes;
    \item[\textit{Stability of Gait}]
    This set of tasks evaluates the overall ability of walking, starting from speed, observing balance and imbalance, pivot turns, the ability of stepping over obstacles and, last but not least, the patient is asked to perform the timed up and go test once on its own and one more time while asked to perform another task (counting, talking, and so on). 
\end{description}
The test can also be performed in its \textit{mini} form, the Mini Balance Evaluation Systems, 
which comprises only 14 of the 36 activities of the original test, coming from sections 3, 4, 5 and 6. 
The total score is 28 (32 if counting separate points for each leg). 

%non riewsco a trovare la classificazione di high risk or low risk a seconda dello score ======
\vspace{0.3cm}
\textbf{Sit To Stand Test} %citazioni ok

The test aims at testing leg strength and endurance in older adults. It is performed using a folding chair with no arms, positioned against a wall and with rubber tips on the legs to prevent it from moving. Also defined as 30 seconds Sit to Stand, the patient is asked to stand up and sit on the chair as many times as they can before the 30 seconds mark, with their arms crossed on their chest and a foot positioned slightly in front of the other for balance reasons. 
The total score is equal to the number of complete, correct stands the subject achieves in the time available. 
A below average number of stands for the patient’s age group indicates a high risk of falls \cite{SittoStand}.
 \vspace{0.5cm} 

\begin{table} [htb!]
    \centering
    \begin{tabular}{|>{\centering}m{3cm} |>{\centering}m{3cm} |>{\centering}m{2.5cm} | >{\centering}m{3cm} |}
        \hline
         \textbf{Age Group}   &  \textbf{Below Average}   &  \textbf{Average}  &  \textbf{Above Average} \tabularnewline
        \hline
        60 - 64	& \textless 14	& 14 to 19 &  \textgreater 19 \tabularnewline
        \hline
        65 - 69	& \textless 12	& 12 to 18 &  \textgreater 18 \tabularnewline
        \hline
        70 – 74	& \textless 12	& 12 to 17 &  \textgreater 17 \tabularnewline
        \hline
        75 – 79	& \textless 11	& 11 to 17 &  \textgreater 17 \tabularnewline
        \hline
        80 – 84	& \textless 10	& 10 to 15 &  \textgreater 15 \tabularnewline
        \hline
        85 – 89	& \textless 8	& 8 to 14 &  \textgreater 14 \tabularnewline
        \hline
        90 – 94	& \textless 7	& 7 to 12 &  \textgreater 12 \tabularnewline
        \hline
    \end{tabular} 
    \caption{Number of average repetitions completed by men}
    \label{tab:STSmen}
\end{table}
 
\vspace{0.5cm} 

\begin{table} [htb!]
    \centering
        \begin{tabular}{|>{\centering}m{3cm} |>{\centering}m{3.5cm} |>{\centering}m{2.5cm} | >{\centering}m{3.5cm} |}
        \hline
         \textbf{Age Group}   &  \textbf{Below Average}   &  \textbf{Average}  &  \textbf{Above Average} \tabularnewline
        \hline
        60 - 64	& \textless 12	& 12 to 17 &  \textgreater 17 \tabularnewline
        \hline
        65 - 69	& \textless 11	& 11 to 16 &  \textgreater 16 \tabularnewline
        \hline
        70 – 74	& \textless 10	& 10 to 15 &  \textgreater 15 \tabularnewline
        \hline
        75 – 79	& \textless 10	& 10 to 15 &  \textgreater 15 \tabularnewline
        \hline
        80 – 84	& \textless 9	& 9 to 14 &  \textgreater 14 \tabularnewline
        \hline
        85 – 89	& \textless 8	& 8 to 13 &  \textgreater 13 \tabularnewline
        \hline
        90 – 94	& \textless 4	& 4 to 11 &  \textgreater 11 \tabularnewline
        \hline
    \end{tabular} 
    \caption{Number of average repetitions completed by women}
    \label{tab:STSwomen}
\end{table}
 
\vspace{0.3cm}
\textbf{Single Leg Stance Test (SLST)} %citazioni messe

The Single Leg Stance (SLS) Test is used to assess static postural and balance control in clinical settings, 
to monitor neurological and musculoskeletal conditions.
This evaluation method is usually used to determine the balance level of people who are at major risk of fall. 
The patient is asked to stand on one leg without assistance, keeping their eyes open and their hands on their hips. 
Once the movement is initiated, a timer is started.
If the equilibrium is lost or one of the hands leaves the hips, the stopwatch is stopped. 
The subject is determined at greater risk of injury from fall, if the balance is kept for less than 5 seconds \cite{SingleLeg}, 
since the movement is significantly impaired by: neurological conditions like multiple sclerosis, Parkinson’s disease, 
Alzheimer’s disease, and dementia; 
stroke; traumatic brain injury; or lower extremity pathology, which usually affect the geriatric population.

\vspace{0.3cm} %citazioni messe
\textbf{Functional Reach Test (FRT)}

The FRT test is comprised of a single activity to evaluate static balance through maximal forward reach from a fixed base of support. It is usually targeted for the elderly and frail adults.
The patient is first asked to stand, and later to extend their arm. Once in position, the subject is requested to reach forward with their lifted arm the most they can standing still, and without losing their balance. 

Once the subject completes the task, the distance reached (measured as the difference of their hand's ending position and their hand's initial position) can be interpreted as a measure of their fall risk if compared to the performance of their peers. 

It has been observed that a greater distance equals to a better balance and decreased fall risk. Thus, in community dwelling elders, a value below 17.5 cm suggests limited mobility skills,
inability to leave the neighborhood without help, and restriction in performing any kind of activity of daily living (ADL) \cite{FRTadl}. Instead, in frail elderly patients, a value below 18.5 cm reach indicates fall risk \cite{FRTfallrisk}. Based on a Canada-wide sample of 2,305 elderly people, the median distance was 29 cm in cognitively unimpaired subjects \cite{FRTnonfrail}.

Some research \cite{FRTreasearch1}  \cite{FRTreasearch2} found that a decreased spinal flexibility and the movement strategy affects the distance reached, questioning the ability of the test to differentiate elderly non-fallers and fallers. A research \cite{FRTtruckmobility} also noted that trunk mobility has a greater contribution to the test than the centre of pressure displacement. 

\vspace{0.3cm}
\textbf{Tinetti Performance Oriented Mobility Assessment (POMA)}

The Tinetti Scale is a test comprised of different activities, developed for the aim of assessing the older adult’s gait and 
balance abilities. Each activity is graded on a scale of 0 to 2 points, according to a given criteria. 
Overall, the total score (gait and balance tests) is 28 points, with 
\begin{itemize}
    \item below 19, high fall risk;
    \item 19 to 24, medium fall risk;
    \item 25 to 28, low fall risk.
\end{itemize}
For the balance tests, the movements tested are:

\vspace{1cm} 

\begin{longtable} [!htb]{|>{\centering}m{5.5cm} |>{\centering}m{5.5cm} |>{\centering}m{1.5cm} |}
        \hline
         \textbf{Tasks}   &  \textbf{Metrics}   &  \textbf{Points}  \tabularnewline
        \hline
         Sitting balance  &  Leans or slides in  chair  &  0 \tabularnewline
                          &  Steady  &  1 \tabularnewline
        \hline
         Rising  &  Unable without help  &  0  \tabularnewline
                 &  Able with help  &  1  \tabularnewline
                 &  Able without help  &  2  \tabularnewline
        \hline
         Attempt to rise  &  Unable without help  &  0  \tabularnewline
                          &  Able with more than 1 trial  &  1  \tabularnewline
                          &  Able in 1 trial  &  2  \tabularnewline
        \hline
         Immediate standing balance (first 5 seconds)  &  Unsteady (swaggers, moves feet, trunk sway)  &  0  \tabularnewline
                                                       &  Steady but uses walker or other support  &  1  \tabularnewline
                                                       &  Steady without walker or other support  &  2  \tabularnewline
        \hline
        Standing Balance & Unsteady & 0 \tabularnewline
                         & Steady but wide stance (median heels more than 4 inches apart) and uses cane or other support & 1 \tabularnewline
                         & Narrow stance without support & 2 \tabularnewline
        \hline
        Nudged &  Begins to fall  & 0 \tabularnewline
        (subject at maximum position with feet as close
        together as possible, & Staggers, grabs, catches self & 1 \tabularnewline
        examiner pushes lightly on subject’s
        sternum with palm of hand 3 times) & Steady & 2 \tabularnewline
        \hline
        Eyes Closed (at maximum position of item 6) & 
Unsteady & 0 \tabularnewline
& Steady  & 1 \tabularnewline
\hline
Turning 360 Degrees & Discontinuous steps & 0 \tabularnewline
 & Continuous steps &1 \tabularnewline
 & Unsteady (grabs, staggers) &0 \tabularnewline
 & Steady &1 \tabularnewline
\hline
Sitting Down &
Unsafe (misjudged distance, falls into chair) &0 \tabularnewline
& Uses arms or not a smooth motion &1 \tabularnewline
& Safe, smooth motion &2 \tabularnewline
\hline
    \caption{Tinetti Scale Tasks}
    \label{tab:TinettiScaleTasks}
\end{longtable}



\vspace{1cm} 

\begin{longtable} [h!]{|>{\centering}m{5.5cm} |>{\centering}m{5.5cm} |>{\centering}m{1.5cm} |}
        \hline
         \textbf{Tasks}   &  \textbf{Metrics}   &  \textbf{Points}  \tabularnewline
        \hline
        Initiation of Gait (immediately after told to “go”) & 
Any hesitancy or multiple attempts to start &0 \tabularnewline
& No hesitancy &1 \tabularnewline  
        \hline
         Step Length and Height (Right swing foot) &
            Does not pass left stance foot with step &0 \tabularnewline
            &Passes left stance foot &1 \tabularnewline 
            &Right foot does not clear floor completely With step &0 \tabularnewline
            & Right foot completely clears floor &1 \tabularnewline   
            \hline
        \vspace{0.5cm} Step Length and Height (Left swing foot) &
        Does not pass right stance foot with step \vspace{0.5cm} &0 \tabularnewline
        &Passes right stance foot \vspace{0.5cm} &1 \tabularnewline 
        &Left foot does not clear floor completely With step \vspace{0.5cm} &0 \tabularnewline
        & Left foot completely clears floor &1 \tabularnewline 
        \hline
        Step Symmetry &
Right and left step length not equal (estimate) &0 \tabularnewline
& Right and left step length appear equal &1 \tabularnewline 
        \hline
         Step Continuity &
 Stopping or discontinuity between steps &0 \tabularnewline
 & Steps appear continuous &1 \tabularnewline
        \hline
        Path &  \vspace{0.5cm} Marked deviation \vspace{0.5cm}  &0 \tabularnewline
  (estimated in relation to floor tiles, 12-inch diameter; & Mild/moderate deviation or uses walking aid &1 \tabularnewline
 observe excursion of 1 foot over about 10  ft. of the course) & Straight without walking aid &2\tabularnewline
        \hline
        Trunk &
 Marked sway or uses walking aid \vspace{0.5cm} &0 \tabularnewline
 & No sway but flexion of knees or back or
 Spreads arms out while walking \vspace{0.5cm} &1 \tabularnewline
 & No sway, no flexion, no use of arms, and no
 Use of walking aid &2 \tabularnewline
 \hline
 Walking Stance
 & Heels apart \vspace{0.5cm} &0 \tabularnewline
 & Heels almost touching while walking \vspace{0.5cm} &1 \tabularnewline 
 \hline
    \caption{Tinetti Scale Tasks Gait Tasks }
    \label{tab:TinettiScaleGait}
\end{longtable}


\section{Relevance of the problem in the context of computer engineering}

Although the problem of fall prevention is already discussed in the medical field, the biological solutions to the problem include the use of the tests discussed above for the determination of the gait asymmetries 
that influence the patient, to evaluate, according to some criteria, if the patient might be at risk. After the examination, the patient is either adviced to perform exercises that can increase their mobility, or to use devices that can prevent sudden falls.
Apart from the biological tests, that are dependent on the medical staff's considerations, computer engineering can lead to advances in the solution to the problem, by making use of different methodologies such as wearable sensors and machine learning. 
Wearable sensors allow for a continuous moniroting of the patient's movements, and include accelerometers, gyroscopoes, magnetoscopes and pressure insoles, whereas machine learning can bring a new insight on the problem by highlighting problem-specific features
that are mostly associated with an inbalanced gait and risk of falls. By giving a new perspective on the problem, both of these means can help the medical staff in the identification of the predominant factors that make patients at risk.
Since the application is related to the medical field, the machine learning algorithms used need to be of the explainable or interpretable type. Using more complex (such as Deep Learning) models or those who hide their solution (such as Neural Networks) is not an option, mainly for two reasons:
the available data is usually limited because it should be acquired from patients who are not always willing to share their information (privacy reasons), and because the medical staff has to be aware of the criteria used by the model to make the prediction, since the results need to be validated and understood by the community to be effective.
One problem with said methods is that the subjects should feel at ease while making use of them. Therefore, an optimal solution to the problem is one that makes use of technologies that are not defined as burdens by the patients who need to make use of them. Therefore, even in the use o0f
sensors, those need to be configured so that the subject feels no discomfort in wearing them. One available solution that has the least amount of discomfort is make of pressure insoles that can be worn inside of shoes to monitor in real time the walking motion of the patient, and assign a risk score
when the signals are analyzed. 
One complication of this approach emerges when considering that the persons who are usually the target of the analysis are mostly elderly or affected by impairements.
Thus, a continuous analysis is probably not feasible if the monitored subject forgets to wear their sensors of forgets to charge them (since the whole system should be equipped with a battery).
Despite this evident limitation, the solution remains valid as it can be used for monthly or yearly checks to determine if the patient has improved their gait, or has become more at risk.
