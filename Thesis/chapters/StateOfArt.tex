%cap2 stato dell'arte su metodi ingegneristisci per la prevenzione delle cadute, mediante utilizzo di sensori indossabili, possibili dataset disponibili
\section{Introduction}
This chapter aims at 
%The "state of the art" refers to the current state of knowledge, technologies, and research on a specific topic or field, particularly in academic and scientific contexts. This section provides a comprehensive and up-to-date overview of the literature and key developments related to fall detection and risk assessment in older adults using wearable sensors.
\section{Definition of the Research Field}
%This section starts by defining the scope of the research field and introducing the main concepts and terms used in the area of interest.
\section{Literature Review}
%The literature review presents a synthesis of the most significant and influential works published on the topic. The studies are organized logically, such as in chronological or thematic order, highlighting the main findings and outcomes from previous research.
A first review on the state of the art was proposed by \cite{FallRiskAssessment_WearableSensors_SystematicReview}. The paper aims at identifying the most common technologies utilized to recognize the risk of fall by evaluating selected articles, from 2002 to 2019, based on a predefined criteria,to detect a fall and to classify the elderly as "faller" and "not faller".  
Focusing on fall risk prediction articles, the authors highlighted that the most common wearable sensor used is the accelerometer, together with the gyroscope. These devices were used in the 17.2\% of the publications to capture gait data. In particular, the best locations to identify the level of risk are the lumbar and waist areas, which were the most popular among the articles considered. Moreover, their sampling rate were mostly 50Hz or 100Hz, although the parameters ranged from 4Hz to 256 Hz. 
Lastly, the methodology most used is the feature extraction, compared to machine learning alone. \\

Predicting falls is a task most centered around the elderly who should be, first of all, classified at risk of falling. An experiment carried out by Rivolta et al. \cite{TinettiScale}, studied the data acquired from 90 participants (79 elderly and 11 healthy volunteers) located into two rehabilitation centers in Northen Italy to asses their risk of falling using the data captured by only one accelerometer attached to the waist. The aim of the study was to create an automated system that could be used at home to predict the risk of a fall.
The study is based on the Tinetti Scale; therefore, the tasks that the participants were asked to perform were both the Gait and Balance activities that comprise the test. Later, this data was used to first study the features obtained by pre-processing the data to establish the differences between high risk fallers and low risk fallers. Secondly, the features were selected using a linear regression to determine those related to the Tinetti Scale; Thirdly, the reduced feature set was used to train two classification models (linear model and artificial neural network) to automatically detect those affected by the risk of fall; Lastly, the scores from the Tinetti Scale were tested to determine which ones were more affected by misjudgements. 

As for the first task, the authors found out that 12 out of 21 features were different among the two groups, while BMI is not significative to determine the difference between high and low risk.
Moreover, as regards balance, the Standing UP Duration, Balance Recovery after nudge and regularity of standing were highly different; whereas the 360 degree turn and the sitting down activities do not differ between the two groups.
As for gait, instead, all the features differ (Step Frequency, Step Regularity - vertical axis, Surrogated Step Height and Trunk Lateral Sway) except Step Regularity computed on the horizontal axis. 

For the second task, Rivolta et al. used LASSO with parameter $\lambda$ = 0.7 to reduce the set to 9 features: Triangle duration along the Anterior/Posterior axis (determined intersecting two straight lines having the maximum and minimum slopes, with the
AP baseline before and after standing up),  Immediate Standing Unbalance for the Standing Up Tinetti Task, Lateral Oscillations of the participant's body when nudged by the specialist, Sample Entropy for Task 5 of the test, and Step Regularity on the vertical axis and Trunk Lateral Sway for the walking task, BMI, Gender and Age.

The classification task involves one regression and one binary classification. The regression performance was measured by the Pearson's correlation coefficient while classification performances were assessed using sensitivity and specificity.
First, the linear regression was trained on the parameters of the Tinetti scores in the training set. Then the artificial neural network was used to understand if it would have a lower number of misclassifications compared to the first approach. 
First, a standard linear regression was used to fit the LM parameters
on the Tinetti scores of the training set. The model was applied on the
test set and the estimated scores were dichotomized at 18 as in the
regular Tinetti test.
Second, an ANN was used to verify whether a non-linear transformation of the feature set could provide a lower misclassification error.
For the regression, the obtained sensitivity was of 0.71 and the specificity was 0.81.
For the artificial network, the obtained sensitivity was of 0.86 and the specificity was 0.90 on the test set.

For the last task, to assess the importance of a misjudgement of the parameters of the Tinetti test in deciding the fall risk, each element was changed in value (choosing from the possible value ranges) and the authors assessed how many participants changed class (from high risk to low risk and viceversa).  

For the balance parts, the most important features were standing balance
with eyes closed, nudge and standing balance with eyes opened, while the less important was sitting balance.
For the gait parts, path related to the floor, trunk sway and walking stance showed the highest classification error when changed, together with walking stance, a determining factor to classify the patients as low risk when changed to 1.
For both parts, a wrong standing balance led to the highest error rate.

Another article used random forest algorithms to classify frailty and falling history in seniors using plantar pressure measurement insoles \cite{RandomForestInsoles}. This study aimed at providing an analysis on the gait of the elderly using insoles able to capture seven signals coming from each foot. Then, the data has been processed to obtain the determining features of the gait, obtaining up to 182 features later analyzed using random forest algorithms. 
The dataset used for the study is comprised of 774 elderly people from Japan. 
The experiment involved different tasks:
\begin{itemize}
    \item A 45s standing balance test;
    \item 20m walking trail (10m n forward direction, then a turn and another 10m in the backward direction)
\end{itemize}

A total of 203 subjects were considered frail by using the Kihon checklist, while 45 subjects who could not remember whether they had fallen in the 12 months prior to the study were excluded. Another 17 subjects were excluded from the analysis due to errors during the collection of the data. Overall, 712 subjects were used for the study. 
A total of 5 categories for features were extracted from the walking data:
\begin{description}
    \item[Frequency Domain Analysis] A fast Fourier transform was performed after the output of the 14 sensors was summed at each sampling point to integrate the temporal information on only one time series.This category includes four extracted features.
    \item[Peak analysis and area under the curves]First, three parameters were extracted for each isolated step and each sensor: maximum pressure,the time at which this maximum pressure occurred relative to the total stance time, the area under the pressure curve was extracted for each isolated step and for each sensor. Second, the four following data features were calculated for each trial and each of the three parameters: the average of all the left foot steps, the standard deviation of all the left foot isolated steps, the standard deviation of all isolated steps of both feet, and the left and right foot average difference. Thus, this results in 84 extracted features in this category.
    \item[1-foot COP trajectory analysis] The COP trajectory was computed for each stance phase of each isolated step. First, the following 13 parameters were extracted: the minimum and maximum values on x and y, x and y coordinates at the double to single stance and single to double stance points, x coordinates of COP at the y coordinates of the center midfoot and center forefoot sensors, respectively; the range of variations on x and y; and the center of pressure excursion index, calculated as the ratio of COP trajectory excursion on the distance between the lateral and medial forefoot sensors. Second, the four following data features were calculated for each trial and each of the 13 parameters: the average of all the left foot isolated steps, the standard deviation of all the left foot isolated steps,the standard deviation of all isolated steps of both feet, and the left and right foot average difference. This category included 52 extracted features.
    \item[Gait phase analysis] The following two parameters were computed for each isolated step: stance phase duration and percentage of double support duration relative to the whole stance phase. Then, the following four features were extracted for each trial and each of the two parameters: average of all the steps from the left foot, standard deviation of all the steps from the left foot, the standard deviation of all isolated steps of both feet, and the left and right foot average difference. This subcategory includes a total of 8 extracted features.
    \item[Wavelet analysis] For each stance phase of each isolated step, the envelope of the 7-sensor of the left foot was computed. This category of features is based on the characteristics of the two waves, which characterize the plantar pressure pattern during the stance phase. First, the following 15 parameters were computed: the distance between the first and second peaks, height of the first peak, height of the second peak, height of valley, the difference between the heights of the peaks, ratio of the height of the first peak to one of the valleys, ratio of the height of the second peak to one of the valleys, difference between these two ratios, width of the first peak, width of second peaks, difference between these two widths, slope rate from the starting point of the stance phase to the first peak, slope rate from the first peak to valley, slope rate from valley to the second peak, slope rate from the second peak to the endpoint of stance phase. Second, the average and standard deviation of all the steps from the left foot were calculated, resulting in the extraction of two features for 15 parameters. This category included 30 extracted features.
\end{description}

Lastly, random forest models were trained to classify the frail from the healthy using scikit-learn. The training procedure was set so that each forest model was composed of 200 decision trees. Each tree is built by successfully splitting its nodes until the Gini impurity score equals zero until all data points in the leaf nodes correspond to the same class.
Models showed low classification performances (average balanced accuracy: 0.57±0.05, weighted F1-score: 0.556±0.034).
Then, the models were validated using a 5-fold cross-validation procedure.  Alternatively, random forest models were constructed and tested using a nested cross-validation procedure. 
The random forest classifiers showed an average balanced accuracy of 0.75±0.04 and an average weighted F1-score of 0.77±0.03 for the recognition of frail vs. non-frail subjects.

\section{Data Availability}
A common problem in applications involving data acquired from any sort of patient is the data availability. Since new systems aim at improving the quality of therapies, make them easily accessible, or help professionals in their everyday activities, these objectives need data directly acquired from those affected to define and create systems that work in the correct way. This involves problems related to the privacy of the patients who make their information accessible to only the part that expressively asked for the data. This usually comes with a (rightful) prohibition of distribution of the fore mentioned data, which usually includes private information about the patients involved. The downside is a limitation of the progress of the research. 
A few public datasets containing the anonymized data are available for everyone to use; other can be accessed upon request, and others are not publicly available. The focus of this thesis revolves around data that has not been processed prior to the release of the dataset to the scientific community. This information is referred to as raw data.
\subsection{Public Datasets}
\subsection{Datasets available upon request}
A first dataset available upon request is \textit{The Smart Insole Dataset} \cite{TheSmartInsoleDataset}. This dataset, available upon request, is comprised of raw data coming from a pair of pressure sensors insoles.
The participants, 29 persons of different age groups and health conditions (healthy adults, elderly, Parkinson’s disease patients), performed two sets of tests. The first group is the control group, in which healthy adults aged between 20–59 years were included; the second group includes elderly citizens, in which people above the age of 60 years were included; the third group relates to Parkinson’s disease patients, irrespective of their age. 
The data is acquired from \textit{Moticon SCIENCE} insole sensors, with a sampling rate of 100Hz. The total number of features is 50 (with 25 coming from the right foot and 25 coming from the left foot), plus a timestamp. The raw data is measured in: 
\begin{itemize}
    \item timestamp, in \(ms\);
    \item pressure points (1 to 16) in \(N/cm^2\);
    \item acceleration on the \(c, y, z\) axes in \(g\);
    \item The angular rate \(\omega x, \omega y, \omega z\) in \(dps\);
    \item The computed center of pressure in the \(x,y\) coordinates;
    \item The total force \(N\).
\end{itemize}

The tests to measure gait information involved the Walking Straight and Turn test and a modified version of the Timed Up and Go test, which are made up of the following activities: for the Walking Straight and Turn test, participants were requested to walk in a straight route for 10m starting from a standing position. At the end of the 10m route, they turned 180° and returned to the starting position. The test was repeated two times and at three different walking speeds, slow, normal, and fast, as perceived by each participant; for the modified Timed Up and Go Test, participants were asked to rise from the chair, and immediately start walking straight for 10m. Then, after a turn of 180°, they were instructed to return to their chair, and sit down. As for the first test, this task was also repeated twice.

The labeling of the recorded data was performed following a two-level annotation process. In the first level of annotation, data were described as activities of daily living (ADLs) that included 12 different labels (Tab \ref{tab:SmartInsoleADL}). The second level of annotation was focused on labeling the characteristic events of a gait cycle (Tab \ref{tab:SmartInsoleGait}), for both the right and left leg. 
\vspace{0.5cm} 

\begin{table} [h!]
    \centering
    \begin{tabular}{|>{\centering}m{2.5cm} |>{\centering}m{2.5cm} |>{\centering}m{6.5cm} |}
        \hline
         \textbf{Label}   &  \textbf{Activity}   &  \textbf{Description}  \tabularnewline
        \hline
        STD	& Standing	& Standing with subtle movements \tabularnewline
        \hline
        STE	& Standing Eyes closed & Standing with eyes closed \tabularnewline
        \hline
        WAL &	Walking &	Normal walking \tabularnewline
        \hline
        WAS	& Walking Slow	& Walking in a slow rhythm \tabularnewline
        \hline
        WAF	& Walking Fast	& Walking in a fast rhythm \tabularnewline
        \hline
        SCH	&Sit on chair &	Sitting on a chair \tabularnewline
        \hline
        CHU	& Chair up&	Getting up from a chair \tabularnewline
        \hline
        SIT	& Sitting	& Sitting with subtle movements \tabularnewline
        \hline
        TUR	&Turning&	Turning 180 degrees at a normal speed at the end of the 10 m aisle \tabularnewline
        \hline
        TUS	& Turning Slow	& Turning 180 degrees at a slow speed at the end of the 10 m aisle \tabularnewline
        \hline
        TUF	& Turning Fast&	Turning 180 degrees at a fast speed at the end of the 10 m aisle \tabularnewline
        \hline
    \end{tabular} 
    \caption{ADL labels of The Smart Insole Dataset}
    \label{tab:SmartInsoleADL}
\end{table}
\vspace{0.5cm} 

\begin{table} [h!]
    \centering
    \begin{tabular}{|>{\centering}m{2.5cm} |>{\centering}m{2.5cm} |>{\centering}m{6.5cm} |}
        \hline
         \textbf{Label}   &  \textbf{Activity}   &  \textbf{Description}  \tabularnewline
        \hline
        HES	& Heel Strike & Heel contacts the floor \tabularnewline  
        \hline
        HER	& Heel Rise	& Heel rises of the floor \tabularnewline     
        \hline
        TOF	& Toe Off& Toe leaves the floor \tabularnewline
        \hline
        FOF	& Foot Flat & Foot is flat on the floor; both heel and toe contact the floor.\tabularnewline 
        \hline
    \end{tabular} 
    \caption{Gait labels of The Smart Insole Dataset}
    \label{tab:SmartInsoleGait}
\end{table}

\section{Challenges}
%This part identifies gaps in the research field and challenges that remain unresolved. Addressing these gaps can justify the relevance of your own research and demonstrate how it contributes to advancing the discipline.
\section{Emerging Trends}
%This section describes any new trends or research directions that are emerging in the field. This may include the introduction of novel technologies, methodologies, or conceptual approaches.
\section{Conclusion}
%The conclusion summarizes the key points presented in the "state of the art" section and emphasizes the significance of continuing research in the field. It provides an overview of how your own work fits into the context of the existing literature and how it can contribute to addressing the identified gaps.
